% !TEX root = ../thesis.tex
%
\chapter{Introduction}
\label{sec:intro}

\section{Motivation and Problem Statement}
\label{sec:intro:motivation}

\section{Results}
\label{sec:intro:results}

\section{Thesis Structure}
\label{sec:intro:structure}
The thesis is structured as follows. Firstly in \textbf{\treft{sec:concepts}} we give a quick overview over the topics and techniques used in this work including a short introduction to the inner workigs of \glspl{cnn}. In \textbf{\treft{sec:related}} we list related works that tackle different subtasks of our method. Next in \textbf{\treft{sec:pipeline}} we explain in depth our pipeline and report extensive test results in \textbf{\treft{sec:results}}. Finally in \textbf{\treft{sec:application}} we show how this method can be practically applied in the real world and give a short conclusion and outlook in \textbf{\treft{sec:conclusion}}.

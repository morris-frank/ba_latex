% !TEX root = ../thesis.tex
%
\chapter{Introduction}
\label{sec:intro}
Today more and more work processes can be supported, or replaced through intelligent machines. Computer Vision proves itself capable of understanding images and video on a near-human level or even outperforming normal human capabilities in some visual tasks. One reason intelligent machines are absolutely needed today and in the future are the enormous masses of data the \textit{digital natives}, companies or scientists generate and store. Independent of the performance difference between computer and human, it its often impossible to work through catalogs or databases of millions or more images, music recordings, scientific data records et cetera. To automate such tasks does not only relieve from monotonous work but also enables humans to achieve more in less time.\\
At the very latest with the development of analog and digital mass storage systems scientists began to base their research on findings from mass data. No different scientists of the humanities move their work into a digital form. Being able to quickly look up any painting as a digital image or shifting through unique, in some archive locked, manuscripts from the middle ages makes drawing connections and research considerable easier. As we can assume these databases to grow or already have grown over any humans capabilities, digital humanities are a prime example for the application of visual learning and automation.

\section{Problem Statement}
\label{sec:intro:motivation}
For this thesis we approach one of possible applications of Computer Vision in the digital humanities, in particular art history. We assume a historian doing work on a huge set of digitized images $\mathbb{I}$. Those could be scans of medieval writings, architecture sketches or photos of historic paintings. The images are assumed to be not at all annotated meaning we do not have any metadata containing e.g. the creator.

Now the historian wants to compare all images $I = \{i_1, i_2,\dotsc,i_n\} \subset \mathbb{I}$ conjunct in-that they all share some visual pattern $p$, may it be they depict the same person or show all a similar scenery. We want to provide the user of this database a method to automatically generate the set $I$. This should be done giving a image $\tau$ which the historian states for, that it contains $p$. Therefore we need a function $C(i)$ which can decide for a image $i\in\mathbb{I}$ whether it possesses $p$ and therefore is part of the seeked set $I$. The only data we can use to adjust the operator $C$ is the sample image $\tau$.\\Proposing one approach for this $C_\tau$ and a generator function $G(\tau) = C_\tau$ is the problem of this thesis.

\section{Thesis Structure}
\label{sec:intro:structure}
The thesis is structured as follows. Firstly in \textbf{\treft{sec:concepts}} we give a quick overview over the topics and techniques used in this work including a short introduction to the inner workings of \glspl{cnn}. In \textbf{\treft{sec:related}} we list related works that tackle different subtasks of our method. Next in \textbf{\treft{sec:pipeline}} we explain in depth our pipeline and report extensive test results in \textbf{\treft{sec:results}}. Finally in \textbf{\treft{sec:application}} we show how this method can be practically applied in the real world and give a short conclusion and outlook in \textbf{\treft{sec:conclusion}}.
